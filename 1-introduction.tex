\section{Introduction}

The rapid advancement of the Internet of Things (IoT) has transformed the concept of home automation, making smart homes increasingly popular \cite{yar2021edge}. Conventional control methods primarily rely on mobile applications or voice commands. While effective, these methods have limitations. Mobile apps require the user to carry a device and navigate through menus \cite{stolojescu2021iot}, while voice commands can be intrusive and unreliable in noisy environments \cite{kurian2023visual}. To overcome these issues, hand gesture recognition has emerged as a promising alternative for intuitive, contactless Human-Computer Interaction (HCI).

Implementing gesture recognition systems usually involves complex deep learning models. Traditionally, these models are deployed on high-performance cloud servers to handle the computational load. However, cloud processing introduces latency, dependency on internet stability, and privacy concerns. A more practical approach for smart home control is local processing directly on embedded devices like the Raspberry Pi. Yet, deploying complex gesture recognition models on these resource-constrained environments presents a significant computational challenge compared to standard desktop processors.

Existing research often utilizes Convolutional Neural Networks (CNN) for static hand pose classification. However, static poses lack the expressiveness required for complex commands. Gestures that involve temporal movement sequences offer a more natural interaction experience but require architectures capable of handling time-series data, such as Long Short-Term Memory (LSTM) networks \cite{sengupta2020review}.

This paper proposes a real-time smart home control system based on sequential hand gesture recognition, utilizing a Raspberry Pi 5 strictly as the primary computational unit. We employ the MediaPipe framework for efficient hand landmark extraction and a custom LSTM model for sequence classification. To ensure real-time performance on the Raspberry Pi hardware, the model is optimized using TensorFlow Lite 8-bit quantization. Furthermore, this study provides a comprehensive computational analysis of the software pipeline, evaluating the model's reliability through a detailed confusion matrix and standard machine learning classification metrics, including Precision, Recall, and F1-Score. Ultimately, this research aims to demonstrate an optimized, high-accuracy software architecture suitable for resource-constrained smart home systems.