\section{Introduction}

The rapid advancement of the Internet of Things (IoT) has transformed the concept of home automation, making smart homes increasingly popular \cite{yar2021edge}. Conventional control methods primarily rely on mobile applications or voice commands. While effective, these methods have limitations: mobile apps require the user to carry a device and navigate through menus \cite{stolojescu2021iot}, while voice commands can be intrusive and unreliable in noisy environments \cite{kurian2023visual}. To overcome these issues, hand gesture recognition has emerged as a promising alternative for intuitive, contactless Human-Computer Interaction (HCI).

Implementing gesture recognition systems usually involves complex Deep Learning models. Traditionally, these models are deployed on high-performance cloud servers to handle the computational load. However, cloud-based processing introduces latency, dependency on internet stability, and privacy concerns. A more efficient approach is Edge Computing, where processing is performed locally on the device. Yet, deploying dynamic gesture recognition models on edge devices like the Raspberry Pi presents a significant challenge due to limited computational resources compared to desktop GPUs.

Existing research often utilizes Convolutional Neural Networks (CNN) for static hand pose classification. However, static poses lack the expressiveness required for complex commands. Dynamic gestures, which involve temporal movement sequences, offer a more natural interaction experience but require architectures capable of handling time-series data, such as Long Short-Term Memory (LSTM) networks \cite{sengupta2020review}.

This paper proposes a real-time smart home control system based on dynamic hand gesture recognition using a Raspberry Pi 5 as the edge processing unit. We employ the MediaPipe framework for efficient hand landmark extraction and a custom LSTM model for gesture classification. To ensure real-time performance on edge hardware, the model is optimized using TensorFlow Lite 8-bit quantization. The system is integrated with ESP8266 microcontrollers to wirelessly control electronic appliances, offering a complete, low-latency, and contactless automation solution.
