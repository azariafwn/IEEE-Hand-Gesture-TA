\section{Conclusion}
\label{sec:conclusion}

This study successfully developed a distributed gesture-based control system leveraging a Stacked LSTM architecture on the Raspberry Pi 5 edge platform. Through 8-bit quantization using TensorFlow Lite, the model achieved significant efficiency gains, reducing runtime memory consumption by 97.7\% and model size by 89.7\% without compromising essential classification accuracy. The system demonstrated robust generalization capabilities across multiple subjects, operating in real-time with an average processing speed of 23.83 FPS and a low edge inference latency of 28.82 ms.

Robustness analysis identified a critical trade-off between visual detail and computational load. Experimental results confirmed that a camera resolution of 480p at a user distance of 50 cm represents the optimal configuration, balancing sufficient feature detail for high accuracy (98.75\%) with the responsiveness required for seamless interaction. Conversely, excessive distances (70 cm) were found to degrade performance due to the loss of landmark granularity.

Overall, the proposed architecture delivers a highly responsive user experience with an average total end-to-end response time of 134.52 ms. A key finding of this research is that the primary performance bottleneck is no longer the edge computing capability, but rather the stability of the wireless network (Wi-Fi), which contributed 78.58\% to the total latency with high jitter. Future work should focus on implementing low-latency protocols such as MQTT or dedicated wireless channels to mitigate this network-induced variability.