\section{Discussion}
This section analyzes the impact of environmental variations and identifying system bottlenecks.


\subsection{Impact of Environmental Variations}
To validate the system's robustness in real-world scenarios, comprehensive testing was conducted across varying environmental conditions, specifically focusing on user distance, lighting intensity, and camera resolution. The detailed experimental results regarding accuracy and latency for each scenario are summarized in Table \ref{tab:env_variations}.

\begin{table}[htbp]
\caption{System Robustness Analysis: Impact of Environmental Variations on Performance}
\label{tab:env_variations}
\centering
\begin{tabular}{l c c c c}
\hline
\hline
\textbf{Condition} & \textbf{Value} & \textbf{Accuracy (\%)} & \textbf{Inference (ms)} & \textbf{Speed (FPS)} \\
\hline
\multicolumn{5}{l}{\textit{A. Impact of User Distance}} \\
Near & 30 cm & 99.0\% & 29.10 & 23.84 \\
\textbf{Optimal} & \textbf{50 cm} & \textbf{99.6\%} & \textbf{28.62} & \textbf{23.76} \\
Far & 70 cm & 98.6\% & 28.73 & 23.87 \\
\hline
\multicolumn{5}{l}{\textit{B. Impact of Lighting Intensity}} \\
Dim & $\approx$ 30 Lux & 98.3\% & 28.85 & 23.96 \\
Normal & $\approx$ 150 Lux & 99.7\% & 28.56 & 23.83 \\
Bright & $\approx$ 400 Lux & 99.2\% & 29.04 & 23.68 \\
\hline
\multicolumn{5}{l}{\textit{C. Impact of Camera Resolution}} \\
\textbf{Low Res} & \textbf{480p} & \textbf{98.7\%} & \textbf{27.19} & \textbf{25.04} \\
High Res & 720p & 99.4\% & 30.45 & 22.61 \\ % Drop FPS di sini jadi argumen kuat!
\hline
\end{tabular}
\end{table}

The experimental results, summarized in Table \ref{tab:env_variations}, reveal that while the system is robust across most conditions, specific configurations yield optimal performance trade-offs.

\begin{itemize}
    \item \textbf{Optimal Range:} At an intermediate distance of 50 cm, the system achieved its peak accuracy of \textbf{99.6\%}. This distance proves to be the "sweet spot" where the hand region is sufficiently large for precise landmark extraction without exceeding the camera's field of view.
    \item \textbf{Distance Robustness:} Performance remained stable even at 70 cm, with only a minor accuracy drop to \textbf{98.6\%}. This indicates that the model generalizes well across varying user positions, maintaining reliable operation even when the user is not in the immediate proximity of the camera.
    \item \textbf{Lighting Independence:} The system demonstrated remarkable resilience to lighting variations. Even in dim conditions ($\approx$30 Lux), the accuracy remained high at \textbf{98.3\%}, only slightly lower than the 99.7\% achieved under normal lighting. This robustness is attributed to the MediaPipe framework, which relies on structural landmark geometry rather than raw pixel intensity, making it superior to traditional color-based segmentation in low-light scenarios.
    \item \textbf{Resolution Trade-off:} The comparison between 480p and 720p presents a critical trade-off between precision and fluidity. While 720p offered a higher accuracy of \textbf{99.4\%}, it increased the computational load, dropping the processing speed to \textbf{22.61 FPS}. Conversely, the 480p resolution achieved a smoother performance of \textbf{25.04 FPS} with a highly competitive accuracy of \textbf{98.7\%}. Consequently, 480p is identified as the optimal configuration, prioritizing interaction fluidity (higher FPS) and lower latency (27.19 ms) over a marginal gain in accuracy.
\end{itemize}


\subsection{Latency Bottleneck Analysis}
A breakdown of the total latency reveals a significant finding:
\begin{itemize}
    \item Edge Efficiency: The optimized LSTM model on Raspberry Pi 5 is highly efficient, contributing only  21.42\% (28.82 ms) to the total latency.
    \item Network Constraint: The Wi-Fi communication between Raspberry Pi and ESP8266 accounts for the remaining 78.58\% (105.71 ms). The high variance in network latency (jitter) suggests that standard HTTP protocols over Wi-Fi are the primary bottleneck for responsiveness, rather than the edge processing capability.
\end{itemize}


