\section{Related Work}

Hand gesture recognition for home automation has been widely explored using various approaches, ranging from computer vision to deep learning. In the context of edge-based implementations, Kurian et al. \cite{kurian2023visual} developed a vision-based home automation system using MediaPipe on a Raspberry Pi. Their system successfully controlled appliances using static poses, such as American Sign Language (ASL) alphabets 'A' and 'B', achieving 95\% accuracy. However, their approach was limited to static gestures, which lack the intuitiveness and complexity required for more natural interaction compared to dynamic movements.

To address the limitations of static recognition, Deep Learning models capable of handling temporal sequences, such as Long Short-Term Memory (LSTM), have been adopted. Gurrala et al. \cite{gurrala2025lstm} demonstrated the effectiveness of LSTM for dynamic gesture recognition in a sign language translation system, achieving an impressive accuracy of 99.6\%. While this study validated the superior performance of LSTM for sequential data, it did not address the challenges of deploying such computationally intensive models on resource-constrained edge devices.

Most closely related to this research is the work by Hidayat \cite{hidayat2025lstm}, who proposed a contactless lamp control system using MediaPipe and LSTM. The system classified gestures into 'On', 'Off', and 'Neutral' states and transmitted commands to an ESP32 microcontroller via HTTP, achieving 97.67\% accuracy with a low latency of 26 ms. Although promising, Hidayat’s implementation relied on a standard laptop for processing, serving primarily as a proof-of-concept. The reliance on a high-performance laptop renders the system unsuitable for embedded smart home nodes due to power and spatial constraints. It left a significant gap in optimizing the model for standalone execution on actual edge hardware. This paper bridges these gaps by implementing an optimized, quantized LSTM model specifically designed to run in real-time on a Raspberry Pi 5 for dynamic gesture control of IoT appliances.
